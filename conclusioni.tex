\large{
Il metodo \textit{Lattice Boltzmann} come visto \`e tuttora utilizzato in tutti gli ambiti inerenti la fluidodinamica proprio per l'algoritmo che, a contrario delle equazioni di Navier-Stokes viste nel capitolo 1, \`e facilmente realizzabile a livello di computazione e ci\`o ha spinto l'attivit\`a di tirocinio formativo a trovare ed analizzare altri aspetti, che potessero affiancare l'utilizzo di questo metodo, da implementare e realizzare.

All'inizio dell'attivit\`a di tirocinio una volta preso nota l'importanza nell'ambito della computer grafica e delle simulazioni dell'utilizzo del metodo Lattice Bolzmann sono stati quindi individuati due obiettivi chiave su cui porre un'attenzione particolare. Primo dei quali, non per importanza, \`e l'aumento della velocit\`a di esecuzione che potesse comunque adattarsi ed essere utilizzato su qualunque tipologia di macchina. Il secondo si concentra invece sulla facilit\`a di utilizzo da parte dell'utente finale, obiettivo messo in primo piano da molti sviluppatori di librerie.

Al termine del tirocinio formativo e con il risultato della progettazione della libreria sono stati raggiunti entrambi gli obiettivi che furono posti tre mesi prima. Il primo, la realizzazione di un software veloce \`e stato raggiunto mediante l'introduzione di due paradigmi di programmazione che puntano proprio alla velocit\`a di esecuzione come visto nei paragrafi 2.2 e 2.3: la programmazione concorrente e parallela, che fa uso di pi\`u thread di esecuzione per suddividere un problema in pi\`u sottoproblemi da svolgere in parallelo introducendo quindi il concetto di multi-tasking e la programmazione generica, che mediante l'utilizzo delle classi e dei metodi templetizzati o "modelli" risolti a tempo di compilazione anzich\`e a tempo di esecuzione migliora anche il riutilizzo del codice e il refactoring.

I risultati ottenuti in questa tesi rappresentano un primo passo esplorativo nella direzione dello studio della velocit\`a computazionale e del confronto di essa a parit\`a di prestazioni della macchina. La velocit\`a computazione infatti tiene conto: delle strutture dati utilizzate, per gli storage e per il reticolo discreto e della quantit\`a di celle, particelle e distribuzioni di velocit\`a di cui \`e composta la griglia, ovvero dalla tipologia di reticolo scelto dall'utente.}
