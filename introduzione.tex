\section{Ringraziamenti}
\label{sec:ringraziamenti}

Non \`e facile andare a ritroso negli anni e rievocare ricordi, emozioni e sensazioni per arrivare ad avere una lista di coloro che mi hanno reso ci\`o che sono e che spero contribuiranno ad essere ci\`o che sar\`o. Desidero ringraziare tutte quelle persone che, con suggerimenti, critiche e osservazioni, hanno fornito un importante aiuto nell'esperienza di questi tre anni appena trascorsi, con la promessa di migliorarmi sempre.	
Ringrazio mia madre, mio padre e mia sorella a cui ho dedicato questi miei tre anni di studio, per aver creduto in me fin dall'inizio di questo percorso. Una dedica, in particolare, va a mia sorella Diandra: nella speranza che tu possa sempre gioire dei miei traguardi come io dei tuoi e con la consapevolezza che ti appogger\`o sempre, anche se con una visione critica e cinica, in qualunque tua scelta futura. Colgo l'occasione per ringraziare coloro con cui ho condiviso lo stesso lungo tragitto di questi tre anni, tutta la rappresentanza studentesca, i miei colleghi e i miei amici. Un ringraziamento speciale va: ad Alessandro, con il quale ho cominciato questo percorso; ad Alex, confidente e grande compagna di avventure; ad Elena e Lorenzo, con i quali ci siamo rialzati; a Francesca, amica di vita. Ultima, ma non per importanza, ringrazio Rea Silvia, il mio lume della ragione, il mio sorriso.
   \begin{flushright}
   	\textit{Luca De Silvestris}
   \end{flushright} 
\newpage
\section{Premessa}
Il lavoro che verr\`a descritto in questa tesi rappresenta la relazione finale di un progetto svolto dal candidato nell'ambito di un tirocinio formativo della durata di tre mesi circa. Il progetto ha riguardato lo sviluppo di una libreria per l'utilizzo del metodo Lattice Boltzmann con l'obiettivo di essere di facile uso e soprattutto efficente a livello di risorse computazionali e di velocit\`a.

Le macro-aree di interesse sono quindi la programmazione C++, la programmazione parallela e concorrente, la programmazione generica e la computer grafica applicata alla fluidodinamica. Tutti i concetti enunciati nella descrizione sintetica qui riportata verranno approfonditi, illustrati, documentati e motivati nel corso della lettura. Alla base dello studio vi \`e la necessit\`a di mostrare l'importanza di questo metodo ormai cos\`i ampiamente utilizzato.\\
La tesi si concentra su questi sei punti chiave:\\ \\
\textbf{1. Stato dell' arte:} breve panoramica sui concetti chiave del lavoro;\\ \\
\textbf{2. Strumenti e metodologie: }un focus sugli strumenti e sulle tecnologie usate per risolvere i problemi e sulle metodologie;\\ \\
\textbf{3. Analisi dei requisiti e problemi:} la definizione degli obiettivi del tirocinio e raccolta dei requisiti progettuali e problemi da risolvere individuati durante le fasi di analisi;\\ \\
\textbf{4. Progettazione e implementazione: }la descrizione dell' architettura del progetto finale, con approfondimento delle scelte progettuali;\\ \\
\textbf{5. Esempi di utilizzo e sviluppi futuri:} considerazioni sulle possibili implementazioni future e esempi di applicazione;\\ \\
\textbf{6. Conclusioni: }considerazioni sui risultati ottenuti.
\vspace{0.5cm}
